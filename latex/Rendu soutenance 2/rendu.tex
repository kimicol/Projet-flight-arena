\documentclass[10pt, titlepage]{report}

\usepackage[utf8]{inputenc}
\usepackage[T1]{fontenc}
\usepackage[francais]{babel}

%Caractères spéciaux

\usepackage{lmodern}
\usepackage{amsmath}
\usepackage{amssymb}
\usepackage{mathrsfs}

\usepackage{eurosym} %insertion signe euro
\usepackage{graphicx} %insertion d'images
\usepackage{fancyhdr} %en-tete et pied de page

\title{\bsc{Rapport de la deuxième soutenance}\\Projet flight arena}
\author{mr cube :\\
Vincent \bsc{Rospini-Clerici},\\
Guillaume \bsc{Rebut}\\
%Nikolas \bsc{Miletic}\\
chef de projet : Arthur \bsc{Remaud}}
\date{7 mai 2015}

\pagestyle{fancy}
\fancyhead{}
\fancyfoot{}
\lhead{Rapport de la deuxième soutenance}
\rhead{Projet flight arena}
\lfoot{mr cube}

\begin{document}

\maketitle
\renewcommand{\contentsname}{Sommaire}
\renewcommand{\chaptername}{Partie}

\tableofcontents

\chapter{Retard/Avance par rapport au cahier des charges}

\section{Prévisions}

\section{Retard}

\subsection{Option changer de touches de commande}

Nous voulions rajouter dans le menu des options la possibilité de choisir ses propres touches sur le clavier pour piloter le vaisseau et faire en sorte que le jeu le retienne. Cependant, nous ignorons totalement comment le faire et aucun site internet ne parle de cela.

Comme Unity, de base, permet au démarrage d'assigner ses touches, et que cela n'est qu'un bonus pas nécessaire, nous avons donc abandonné notre idée de base.\\

\section{Avance}

\chapter{Travail par membre}
Nous allons vous décrire ce que chaque membre de l'équipe mr cube a fait pendant cette deuxième période, avec leurs difficultés rencontrées et les techniques utilisées.

\section{Guillaume Rebut}

\subsection{Ce que Guillaume doit faire pour la seconde soutenance}

\section{Vincent Rospini-Clerici}

\subsection{Différents problèmes rencontrés}

\subsection{Ce que Vincent doit faire pour la seconde soutenance}

\section{Arthur Remaud}

\subsection{Multijoueur en réseau}
Pendant cette période, Arthur a surtout travaillé sur le multijoueur. Il a d'abord essayé de faire un protocole UDP pour relier en LAN (Local Area Network) en s'inspirant du TP que nous avions fait dans les cours habituels lorsque nous avions travaillé sur le protocole TCP, avant de se rendre compte qu'il existait la classe \textit{Network} sur Unity qui simplifie grandement l'élaboration d'un mode multijoueur sur un jeu. Plusieurs tutoriels existent à ce sujet sur internet, Arthur s'en est donc inspiré pour faire un réseau.\\

Maintenant, le joueur peut choisir d'héberger une partie en LAN ou de rejoindre une partie déjà existante. S'il crée sa propre partie, il s'affiche alors dans un coin son adresse IP locale pour qu'il puisse la donner aux joueurs souhaitant le rejoindre. Ces derniers doivent d'abord saisir l'adresse IP pour rejoindre une partie, et ensuite leur vaisseau apparait dans la partie pour commencer à jouer.

Les règles du jeu sont les mêmes que dans le mode un joueur.\\

Le principal problème fut qu'au départ, les joueurs contrôlaient le vaisseau de l'autre joueur et donc devait regarder sur l'autre écran pour jouer. De plus, à trois joueurs, les deux premiers connectés voyaient un seul et même vaisseau qu'il contrôlaient tous les deux pendant que le troisième joueur en pilotait un autre. Le dernier vaisseau quant à lui n'était contrôlé par personne et restait immobile. Au final, ce problème venait de l'assignation des caméras et des scripts aux joueurs lorsqu'ils instanciaient un nouveau vaisseau en arrivant.\\

\subsection{Quaternions}
Il nous avait été demandé lors de la dernière soutenance de modifier les déplacements des vaisseaux en rajoutant des quaternions pour faire les rotations des vaisseaux. Comme c'était Arthur qui s'était chargé des mouvements à la base, c'est lui qui a modifié le code pour mettre à la place des quaternions. Maintenant, les mouvements sont plus réalistes grâce à l'inertie mais cela rend le jeu un peu plus difficile.\\

Nous n'avions pas utilisé cette technique plus tôt car elle laissait les forces de collisions sur le vaisseau qui le faisaient déplacer après avoir touché un immeuble. Nous avons remédié à ce problème en mettant à zéro les déplacements du vaisseau dans la fonction \textit{Update} si le joueur ne bouge pas.\\

Parfois, lorsque le vaisseau tournaient longtemps sur lui-même, il se mettait soudainement à partie dans l'autre sens avant de repartir de la rotation voulue. En effet le quaternion dépassait les 180\textdegree  et donc la rotation s'inversait. En ajoutant un maximum à l'accélération, on a pu régler facilement ce petit imprévu.\\

\subsection{Intelligence artificielle}

Arthur a commencé à faire une intelligence artificielle pour pouvoir jouer contre des vaisseaux contrôlés par l'ordinateur. Pour le début nous voulions faire un algorithme de \textit{pathfiding}. C'est un algorithme qui permet de calculer la trajectoire la plus courte d'un point A à un point B en évitant les obstacles, et donc permet à une intelligence artificielle de se déplacer facilement. Cependant la 3D nous a posé problème, car on ne peut représenter facilement le terrain sous forme de tableau. Il faudrait utiliser un graphe, mais non seulement nous ne savons pas comment le faire, mais en plus il faudrait en faire un différent pour chaque carte, et nous n'avons pas eu le temps.\\

Nous avons donc opté pour une autre technique : devant le vaisseau, nous avons rajouté des cylindres invisibles qui détectent des collisions avec les murs. Lorsque cela se produit, le vaisseau ralentit puis tourne en conséquence pour ne pas rentrer dans l'obstacle. Le vaisseau circule donc entre les bâtiments qu'il détecte et évite. Nous avons remarqué que cela ne fonctionnait pas pour les objets qui avait pour \textit{collider} un\textit{ Mesh Collider}. Nous avons donc rajouté des \textit{Box Collider} à tous les bâtiments et aux limites invisibles du jeu pour qu'il ne fonce pas dedans, tout en veillant à ce qu'il ne crée pas de collision avec les autres vaisseaux.\\

\subsection{Menus}

Le menu a été complété par Arthur. Il a rajouté la sélection de la qualité d'image dans le menu option pour pouvoir choisir en fonction de la performance de son ordinateur les meilleurs graphismes possibles.\\

Il a aussi fait en sorte que le joueur puisse choisir le vaisseau qu'il veut piloter parmi les différents proposés avant de commencer la partie, ainsi que la sélection de la carte en mode un joueur.\\

Au départ, nous voulions faire en sorte que le joueur puisse choisir la répartition des touches de son clavier s'il voulait changer ses commandes pour être plus à l'aise. Cependant, nous ignorons totalement comment le faire et Arthur n'a rien trouvé sur internet pour le faire, même dans la documentation de Unity. Nous avons donc abandonné cette idée, comme nous l'avons dit dans la partie de retard.\\

\subsection{Ce qu'Arthur doit faire pour la prochaine soutenance}

\chapter{Pour la prochaine soutenance}


bla\\ \\ \\ \\ \\ \\ \\

\begin{center}
\includegraphics[height=4cm, width=4cm]{vaisseux_petit.png}
\end{center}

\end{document}