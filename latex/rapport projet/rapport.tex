\documentclass[12pt, titlepage]{article}

\usepackage[utf8]{inputenc}
\usepackage[T1]{fontenc}
\usepackage[francais]{babel}

%Mise en page

\usepackage[top=2cm, bottom=2.5cm, left=4cm , right=3cm]{geometry}
\usepackage{setspace}
\setlength{\parskip}{1ex plus 0.5ex minus 0.2ex}

%Caractères spéciaux

\usepackage{lmodern}
\usepackage{amsmath}
\usepackage{amssymb}
\usepackage{mathrsfs}

\usepackage{eurosym} %insertion signe euro
\usepackage{graphicx} %insertion d'images
\usepackage{fancyhdr} %en-tete et pied de page

\title{\bsc{rapport de projet}\\Projet flight arena}
\author{mr cube :\\
Vincent \bsc{Rospini-Clerici},\\
Guillaume \bsc{Rebut}\\
chef de projet : Arthur \bsc{Remaud}}
\date{16 juin 2015}

\pagestyle{fancy}
\fancyhead{}
\fancyfoot{\thepage}
\lhead{mr cube}
\rhead{Projet Flight Arena}
\lfoot{}
\rfoot{}


\begin{document}
\begin{spacing}{1.5}

\maketitle

\renewcommand{\contentsname}{Sommaire}
\renewcommand{\chaptername}{Partie}

\tableofcontents
\addcontentsline{toc}{section}{Introduction}


\newpage
\section*{Introduction}

Le groupe mr cube, composé de Guillaume Rebut, Vincent Rospini Clerici et Arthur Remaud, est fier de vous présenter son projet : \textit{flight arena}.\\

Ce projet répond à une obligation de l'école \bsc{epita} de faire un projet libre par groupe de quatre. Nous avons donc choisi de faire un jeu de vaisseau dirigé à la troisième personne dans lequel le but est d'éliminer ses ennemis. Le but est de nous apprendre à travailler en équipe, tout en gérant un projet du début à la fin, comportant obligatoirement un cahier des charges, un rapport de projet et plusieurs soutenances.\\

Nous avions pour ce projet l'obligation d'utiliser les langages \textit{Caml} ou {C\#}, avec le moteur Unity pour faire le jeu.\\

Nous étions au départ avec Nikolas Miletic dans le groupe, mais celui-ci est parti en S1\# au cours du mois de janvier. Nous avons alors réparti les tâches de cette manières : Vincent s'occupera principalement des graphismes et du site internet, Guillaume du gameplay et des cartes, et Arthur du code du jeu et des menus.

\newpage
\section{Graphismes et son}

\subsection{Graphismes}

\subsection{Son}

\newpage
\section{Contrôles du vaisseau}

Les contrôles du vaisseau sont les premières choses implémentées dans le jeu. En effet nous avions besoin de déplacements pour faire par la suite pour faire la première carte du jeu, afin d'espacer correctement les obstacles pour que le vaisseau puisse faire des man\oe uvres, et des tirs pour pouvoir gérer les dégâts et la mort des vaisseaux.

\subsection{Déplacement}

Dans un premier temps, Arthur a codé les déplacements des vaisseaux. Pour prendre les touches que le joueur entre, on utilise la fonction \textit{Input.GetKey()} de Unity. Nous avons choisi comme touches pour le départ les même contrôles que dans les jeux \textit{Battlefield}.

Les vaisseaux ont trois types de mouvements : le roulis, rotation du vaisseau selon l'axe longitudinale, le tangage , rotation du vaisseau sur son axe transversal, et le lacet, rotation du vaisseau selon l'axe vertical. Les vitesses de rotations s'inspirent aussi des valeurs des avions : le tangage est plus rapide que le roulis qui est bien plus rapide que le lacet. Nous avons aussi choisi de prendre ces valeurs pour des raisons de gameplay. En effet, il est plus facile de tourner avec le lacet plutôt que d'utiliser la combinaison tangage plus roulis. Il est donc logique de rendre cette dernière man\oe uvre plus rapide en exécution pour récompenser les joueurs les plus talentueux. Nous avons aussi placé la caméra de manière à inciter le joueur à utiliser cette dernière man\oe uvre : le vaisseau n'est pas représenté au milieu de l'écran mais en bas pour donner plus de visibilité. Toujours dans une optique de réalisme et de difficulté, le vaisseau a de l'inertie, lorsqu'il avance et aussi sur les trois types de mouvements, et il est impossible de reculer avec le vaisseau, il faut faire demi-tour.

 Pour avancer, on utilise la touche Z, pour aller vers la gauche on utilise Q, vers la droite c'est D, pour faire une rotation sur la gauche on utilise la flèche gauche, une rotation vers la droite c'est la flèche droite, et enfin pour aller vers le haut on utilise la flèche bas et vers le bas on utilise la flèche haut. Le haut et le bas sont inversés dans les contrôles pour faire comme dans les jeux utilisant un joystick.

Le vaisseau doit donc avancer et tourner sur lui-même pour parcourir les niveaux. Pour cela, nous utilisions tout d'abord respectivement les fonctions \textit{transform.Translate()} et \textit{transform.Rotate()}. L'inertie du vaisseau était gérée par une variable de déplacement qui augmentait à force d'appuyer sur la touche d'accélération, et diminuait dans le cas contraire.

Cependant, les mouvements n'étaient pas très réalistes et il nous avait été suggéré à la fin de la première soutenance d'utiliser les quaternions que propose Unity pour donner une meilleure inertie et donc des déplacements plus crédibles. Cela fut rajouté dans la semaine qui suivit, grâce aux nombreux tutoriels et documentations trouvables sur internet.

Le problème qui se posa avec les quaternions fut en rapport avec les collisions. En effet avant de les ajouter, nous avions mis le \textit{rigidbody} des vaisseaux, soit l'élément qui gère les collisions avec les rebonds et l'inertie engendrée, à zéro. Dans le cas contraire, dès que le vaisseau touchait un obstacle, il rebondissait et partait en vrille, sans que le joueur ne puisse y remédier.

\subsection{Tirs}

Le joueur peut tirer grâce à la touche espace, elle aussi détectée par la fonction \textit{Input.GetKey()}. Unity instancie alors une balle qui est stockée dans un prefab grâce à la fonction qui s'appelle, logiquement, \textit{Instantiate()}. La balle part droit devant elle si elle ne touche aucun obstacle, elle disparait d'elle-même au bout de trois secondes avec la fonction \textit{DestroyObject()}, ce qui lui laisse le temps de traverser la carte, afin qu'il n 'y ait pas trop d'objets en même temps à gérer par Unity, ce qui pourrait ralentir le jeu.

La balle contient un \textit{trigger}, ce qui veut dire qu'elle déclenche un événement lorsqu'elle rencontre un objet qui contient un détecteur particulier. En effet, si un vaisseau détecte une collision avec une balle grâce à la fonction \textit{OnTriggerEnter()}, alors la balle est détruite et le vaisseau perd un point de vie. De plus, si le vaisseau n'a plus de vie, il est alors détruit avec une animation d'explosion. Des sons ont été intégrés à chaque étape : au tir, à la collision et à l'explosion.

\newpage
\section{Gameplay}

\newpage
\section{Création de niveaux}

\newpage
\section{Interfaces utilisateur}

\newpage
\section{Intelligences artificielles}

\newpage
\section{Multijoueurs}

Dans notre jeu, nous voulions absolument qu'il y ait du multijoueur, car le jeu serait ennuyeux si on ne pouvait jouer que contre des ordinateurs. En effet pour un jeu de combat aérien en arène fermée, il fallait pouvoir jouer avec ou contre des amis. Nous avons implémenté deux sortes de multijoueur différents : le multijoueur en écran séparé, et le multijoueur en réseau local.

\subsection{Multijoueur en écran séparé}

\subsection{Multijoueur en réseau}

Le multijoueur en réseau LAN (pour Local Area Network) fut commencé après la première soutenance. Le but était de pouvoir jouer à plusieurs sur plusieurs ordinateurs différents reliés par un réseau local. L'avantage par rapport à l'écran séparé est que le joueur a tout l'écran donc peu mieux voir le jeu, et il n'y a pas de problème de répartition de touches de commandes pour les deux joueurs. C'est Arthur qui s'est occupé de cette partie.

Il a d'abord essayé de faire un protocole UDP pour relier en LAN (Local Area Network) en s'inspirant du TP que nous avions fait dans les cours habituels lorsque nous avions travaillé sur le protocole TCP, avant de se rendre compte qu'il existait la classe \textit{Network} sur Unity qui simplifie grandement l'élaboration d'un mode multijoueur sur un jeu. Plusieurs tutoriels existent à ce sujet sur internet, Arthur s'en est donc inspiré pour faire un réseau.

Un des joueurs héberge la partie, et les autres doivent le rejoindre en utilisant son adresse IP locale. A travers le réseau, il fallait envoyer les données des vaisseaux, leurs déplacements, s'ils tiraient et autre. Les coordonnées vaisseaux sont transmises grâce à l'option \textit{Network view} que l'on peut affecter à un objet. Ainsi, tous les joueurs connectés à cette partie voient le vaisseau généré par le joueur qui se connecte.

L'un des problèmes qui se posa au début, fut que les joueurs ne contrôlaient le vaisseau de l'autre joueur et donc devait regarder sur l'autre écran pour jouer. De plus, à trois joueurs, les deux
premiers connectés voyaient un seul et même vaisseau qu'il contrôlaient tous les deux pendant que le troisième joueur en pilotait un autre. Le dernier vaisseau quant à lui n'était contrôlé par personne et restait immobile. Au final, ce problème venait de l'assignation des caméras et des scripts aux joueurs lorsqu'ils instanciaient un nouveau vaisseau en arrivant.

Un autre problème fut que nous n'arrivions pas à faire disparaitre le vaisseau d'un joueur lorsqu'il se déconnectait de la partie en ligne. En effet la fonction \textit{Network.Destroy()} ne s'appliquait pas au prefab tout entier car il ne contenait pas de networkView. Finalement, nous avons intégré cette fonction dans les scripts qui contrôle les déplacements du vaisseau et de la caméra.

Lorsque l'on démarre le mode réseau, l'antivirus des ordinateurs peut bloquer le jeu, ou tout au moins demander l'autorisation à l'utilisateur de laisser libre la connexion. Cela n'est pas une très grande gêne car une fois que l'on désactive les pare-feu, tout revient dans l'ordre, mais nous ne pouvons pas remédier à ce problème définitivement. Cela ne nous empêche cependant pas de jouer.

\newpage
\section{Site internet}

\newpage
\section*{Conclusion}
\addcontentsline{toc}{section}{Conclusion}

Nous avons adorer faire ce projet tout au long du second semestre. Cela nous a apporté beaucoup de choses, notamment des connaissances des logiciels utilisés comme Unity, Blender \dots Mais au delà de l'aspect technique, cela nous a appris à travailler en équipe, à gérer les emplois du temps, à savoir répartir le travail aux différents membres du groupe en fonction de leurs compétences \dots etc.

\end{spacing}
\end{document}