\documentclass[12pt, titlepage]{article}

\usepackage[utf8]{inputenc}
\usepackage[T1]{fontenc}
\usepackage[francais]{babel}

%Mise en page

\usepackage[top=2cm, bottom=2.5cm, left=4cm , right=3cm]{geometry}
\usepackage{setspace}

%Caractères spéciaux

\usepackage{lmodern}
\usepackage{amsmath}
\usepackage{amssymb}
\usepackage{mathrsfs}

\usepackage{eurosym} %insertion signe euro
\usepackage{graphicx} %insertion d'images
\usepackage{fancyhdr} %en-tete et pied de page

\title{\bsc{rapport de projet}\\Projet flight arena}
\author{mr cube :\\
Vincent \bsc{Rospini-Clerici},\\
Guillaume \bsc{Rebut}\\
chef de projet : Arthur \bsc{Remaud}}
\date{16 juin 2015}

\pagestyle{fancy}
\fancyhead{}
\fancyfoot{}
\lhead{mr cube}
\rhead{Projet Flight Arena}
\lfoot{}


\begin{document}
\begin{spacing}{1.5}

\maketitle

\renewcommand{\contentsname}{Sommaire}
\renewcommand{\chaptername}{Partie}

\tableofcontents
%\addcontentsline{toc}{section}{Introduction}
%\addcontentsline{toc}{section}{Conclusion}

\newpage
\section*{Introduction}

Le projet \textit{flight arena} répond à une obligation de l'école \bsc{epita} de faire un projet libre par groupe de quatre. Nous avons donc choisi de faire un jeu de vaisseau dirigé à la troisième personne dans lequel le but est d'éliminer ses ennemis. Le but est de nous apprendre à travailler en équipe, tout en gérant un projet du début à la fin.\\

Nous avions pour ce projet l'obligation d'utiliser les langages \textit{Caml} ou {C\#}, avec le moteur Unity pour faire le jeu.\\

Nous étions au départ avec Nikolas Miletic dans le groupe, mais celui-ci est parti en S1\# au cours du mois de janvier. Nous avons alors réparti les tâches de cette manières : Vincent s'occupera principalement des graphismes et du site internet, Guillaume du gameplay et des cartes, et Arthr du code du jeu et des menus.

\newpage
\section{Chronologie}

\newpage
\section*{Conclusion}

\end{spacing}
\end{document}